\documentclass[a4paper,12pt]{article}

\usepackage{mathtools}

\begin{document}

\section*{Master Theorem for Divide and Conquer  }

%Displaying  equations between the $ <math stuff> $

%General Equations: 

%$\cos (2\theta) = \cos^2 \theta - \sin^2 \theta$

%$k_{n+1} = n^2 + k_n^2 - k_{n-1}$

%$\frac{n!}{k!(n-k)!} = \binom{n}{k}$

%$\sqrt{\frac{a}{b}}$

General Form of Recurrence Relation

\vspace{5mm}

$T(n) = \alpha T(\frac{b}{\beta}) + f(n)$

\vspace{5mm}
$\alpha \geq 1$  
$\beta > 1$   $f(n) = \theta(n^kLog^pn)$

\vspace{5mm}

The Recurrence Equation provides all of these elements including: 

\begin{itemize}

	\item $log^a_b$
	\item k
\end{itemize}

Based on these two values of $log^a_b$ and k there are 3 Cases:

\vspace{5mm}
Case 1: if $log^a_b > k$ then $\theta(n^{log^a_b})$

\vspace{5mm}
Case 2:  if $log^a_b = k$ then:
	\begin{itemize}
		\item if $p > -1 \cong \theta(n^klog^{p+1}n)$
		\item if $p =-1 \cong \theta(n^{k}loglogn)$
		\item if $p < -1 \cong \theta(n^{k})$
	\end{itemize}
	
\vspace{5mm}
Case 3: if $log^a_b < k$ then:
	\begin{itemize}
		\item if $p \geq 0 \cong \theta(n^klog^{p}n)$
		\item if $p =-1 \cong \theta(n^{k}loglogn)$
		\item if $p < 0 \cong O(n^{k})$
	\end{itemize}

\end{document}